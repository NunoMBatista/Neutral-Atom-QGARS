\documentclass[conference]{IEEEtran}
\IEEEoverridecommandlockouts
% The preceding line is only needed to identify funding in the first footnote. If that is unneeded, please comment it out.
%Template version as of 6/27/2024

\usepackage{cite}


\usepackage{amsmath,amssymb,amsfonts}
\usepackage{algorithmic}
\usepackage{graphicx}
\usepackage{textcomp}
\usepackage{xcolor}
\usepackage{hyperref}

% Add TikZ package and necessary libraries
\usepackage{tikz}
\usetikzlibrary{arrows.meta,positioning,fit,backgrounds,shapes.geometric}

% Add circuitikz package for circuit diagrams
\usepackage{circuitikz}

% Add float control parameters to help with figure placement
\renewcommand{\floatpagefraction}{0.8}
\renewcommand{\topfraction}{0.8}
\renewcommand{\bottomfraction}{0.8}
\renewcommand{\textfraction}{0.1}
\setcounter{totalnumber}{50}
\setcounter{topnumber}{50}
\setcounter{bottomnumber}{50}

% Force LaTeX to place figures earlier
\renewcommand{\dblfloatpagefraction}{0.7}
\renewcommand{\dbltopfraction}{0.8}

\def\BibTeX{{\rm B\kern-.05em{\sc i\kern-.025em b}\kern-.08em
    T\kern-.1667em\lower.7ex\hbox{E}\kern-.125emX}}
\begin{document}

\title{Quantum-Guided Autoencoding for Enhanced Neutral Atom Reservoir Computing in Medical Image Classification\\
{\footnotesize \textsuperscript{*}Note: Sub-titles are not captured for https://ieeexplore.ieee.org  and
should not be used}
}

\author{
\IEEEauthorblockN{Nuno Batista}
\IEEEauthorblockA{\textit{Department of Informatics Engineering} \\
\textit{Faculty of Sciences and Technology, University of Coimbra}\\
Coimbra, Portugal \\
\href{mailto:nunomarquesbatista@gmail.com}{nunomarquesbatista@gmail.com}}

\and
\IEEEauthorblockN{2\textsuperscript{nd} Given Name Surname}
\IEEEauthorblockA{\textit{dept. name of organization (of Aff.)} \\
\textit{name of organization (of Aff.)}\\
City, Country \\
email address or ORCID}
}

\maketitle


\begin{abstract}
We introduce a quantum-classical hybrid 
architecture for medical image classification 
based on neutral atom quantum processors. 
This approach is designed to address the challenges of 
medical imaging, with a particular focus on tasks such 
as polyp detection and classification.
By integrating an autoencoder guided by a quantum 
reservoir, the pipeline learns compact and discriminative 
representations of image data that are also well-suited for quantum reservoir 
computing. To overcome the non-differentiability of quantum 
measurements, we circumvent this `gradient barrier' 
by incorporating a differentiable surrogate model that simulates the 
behaviour of the quantum layer, enabling end-to-end backpropagation. 
The guided training process jointly optimizes for both image 
reconstruction and classification accuracy, ensuring that the latent 
representations are both meaningful and effective for quantum processing. 
In our implementation, image data is encoded as atom detuning parameters 
in a Rydberg Hamiltonian, and quantum embeddings are obtained through expectation 
values. These embeddings are then passed to a linear classifier, 
enabling faster training and inference compared to deep classical networks.
Our experiments show that this method outperforms traditional 
approaches using PCA or unguided autoencoders. We also conduct 
ablation studies to evaluate the impact of quantum and training 
parameters, demonstrating the robustness and flexibility of the 
proposed pipeline for real-world medical imaging applications, 
even in the NISQ era.
\end{abstract}

\begin{IEEEkeywords}
Reservoir Computing, Quantum-Guided Autoencoding,
Neutral Atoms, Autoencoder, Dimensionality Reduction, 
Quantum Machine Learning, Hybrid Quantum-Classical Algorithms, 
Medical Image Classification, Quantum Surrogate Models
\end{IEEEkeywords}

%============================================
% INTRODUCTION
%============================================
\section{Introduction}

\subsection{Background and Motivation}
Advances in medical imaging have significantly improved 
disease diagnosis and treatment planning. For conditions 
like colorectal cancer, early detection of polyps through 
colonoscopy image analysis is critical for reducing mortality \cite{estevaGuideDeepLearning2019a}. 
Deep learning techniques, especially autoencoders, are widely 
used to extract compressed, informative features from 
high-dimensional images for classification and segmentation 
tasks \cite{bengioLearningDeepArchitecturesa}. However, classical neural networks may struggle 
to capture intricate correlations in complex medical data.

Quantum computing offers novel opportunities 
for machine learning, particularly through quantum reservoir 
computing (QRC), where a physical quantum system processes 
classical inputs into high-dimensional nonlinear embeddings 
\cite{tanakaRecentAdvancesPhysical2019,fujiiHarnessingDisorderedQuantum2017}. Recent works show that analog quantum systems, 
such as neutral-atom platforms, can serve as untrained 
reservoirs with rich dynamics for temporal and pattern 
recognition tasks \cite{domingoOptimalQuantumReservoir2022,kornjavcaLargescaleQuantumReservoir2024}. 
In hybrid approaches, a classical encoder compresses 
image data, and a quantum reservoir expands 
the encoded features into a higher-dimensional space, 
potentially boosting classification performance.

A major challenge in such hybrid quantum-classical 
models is the non-differentiability of quantum measurements, 
which obstructs gradient-based optimization. Additionally, 
tuning quantum parameters can suffer from barren plateaus, where 
gradients vanish in high-dimensional Hilbert spaces \cite{mccleanBarrenPlateausQuantum2018}. 
To address this, we introduce a classical neural surrogate 
that emulates the quantum reservoir's input-output behavior. 
This surrogate enables end-to-end training via backpropagation, 
while the quantum system remains fixed and non-trainable.


\subsection{Contributions of This Work}
We propose a quantum-guided autoencoder architecture that 
integrates a classical image encoder with a neutral-atom
quantum reservoir.

A classical surrogate network of the reservoir itself enables 
gradient flow through the whole model during training.

The model is evaluated and compared to classical benchmarks on three 
different datasets: 
\begin{enumerate}
    \item A synthetic dataset of polyp images, generated to simulate realistic medical imaging scenarios.
    \item Real image patches extracted from the CVC-ClinicDB dataset, a well-known benchmark for polyp detection.
    \item A reduced version of the MNIST dataset, containing only the digits 0 and 1, suitable for binary classification tasks.
\end{enumerate}

Our results illustrate the viability of QRC for real-world medical 
tasks and offer a scalable path to hybrid quantum-classical learning,
even in the noisy intermediate-scale quantum (NISQ) era.


%============================================
% BACKGROUND 
%============================================
\section{Background}

\subsection{Principles of Reservoir Computing}
Reservoir computing is a computational framework 
derived from recurrent neural networks (RNNs). It 
involves a fixed, high-dimensional dynamical system—the 
reservoir—that projects input data into a rich feature 
space. Only the output layer is trained, simplifying 
the learning process and reducing computational 
overhead. This approach is particularly effective 
for time-series prediction and pattern recognition tasks.

Mathematically, let \( u(t) \in \mathbb{R}^m \) be the input at time \( t \),
\( x(t) \in \mathbb{R}^n \) the reservoir state, and
\( y(t) \in \mathbb{R}^k \) the output. The reservoir dynamics
and output are given by:
\begin{equation}
    x(t) = f(W_{in} u(t) + W_{res} x(t-1))
\end{equation}
\begin{equation}
    y(t) = W_{out} x(t)
\end{equation}

%Where ff is a nonlinear activation function, WinWin​ and WW are fixed input and reservoir weight matrices, and WoutWout​ is the trained output weight matrix.

Where \( f \) is a nonlinear activation function,
\( W_{in} \) and \( W_{res} \) are fixed input and reservoir weight matrices, 
and \( W_{out} \) is the trained output weight matrix.


% import figure from reservoir_architecture.tex
\begin{figure}[!ht]
    \centering
    %\resizebox{1\textwidth}{!}{%
    \resizebox{\columnwidth}{!}{%
    \begin{circuitikz}
    \tikzstyle{every node}=[font=\normalsize]
    \node [font=\huge] at (17.75,51.5) {};
    
    \draw  (5.5,-1) circle (0.25cm);
    \draw  (5.5,-1.75) circle (0.25cm);
    \draw  (5.5,-2.5) circle (0.25cm);
    \draw  (5.5,-0.25) circle (0.25cm);
    \draw  (7,-0.5) circle (0.25cm);
    \draw  (7.5,-1) circle (0.25cm);
    \draw  (7.75,-1.75) circle (0.25cm);
    \draw  (7.25,-2.25) circle (0.25cm);
    \draw  (8,0) circle (0.25cm);
    \draw  (8.75,-1.75) circle (0.25cm);
    \draw  (9.25,0) circle (0.25cm);
    \draw  (8.75,-0.75) circle (0.25cm);
    \draw  (9.25,-2.5) circle (0.25cm);
    \draw  (8.25,-2.75) circle (0.25cm);
    \draw  (10,-1.75) circle (0.25cm);
    \draw  (10,-0.75) circle (0.25cm);
    \draw [->, >=Stealth] (7.5,-2.25) .. controls (7.5,-2.75) and (7.75,-2.75) .. (8,-2.75) ;
    \draw [->, >=Stealth] (7.5,-2.25) .. controls (8.25,-2.5) and (8,-1.75) .. (8.5,-1.75) ;
    \draw [->, >=Stealth] (8.5,-2.75) .. controls (8.75,-3) and (9,-2.75) .. (9,-2.5) ;
    \draw [->, >=Stealth] (9.5,-2.5) .. controls (10.25,-3) and (10,-2.25) .. (10,-2) ;
    \draw [->, >=Stealth] (9,-1.75) -- (9.75,-0.75);
    \draw [->, >=Stealth] (9.75,-1.75) -- (9,-0.75);
    \draw [->, >=Stealth] (8,-1.75) .. controls (8.5,-1.25) and (8.25,-1.25) .. (8.5,-0.75) ;
    \draw [->, >=Stealth] (8.75,-0.5) .. controls (8.5,-0.5) and (8.75,0) .. (8.25,0) ;
    \draw [->, >=Stealth] (7.25,-0.5) .. controls (7.25,-0.25) and (7.25,0) .. (7.75,0) ;
    \draw [->, >=Stealth] (7.75,-1) .. controls (8.25,-0.75) and (8,-0.5) .. (8,-0.25) ;
    \draw [->, >=Stealth] (9,0) .. controls (9,0.5) and (8.75,0.5) .. (8.25,0) ;
    \draw  (11.5,-0.5) circle (0.25cm);
    \draw  (11.5,-1.25) circle (0.25cm);
    \draw  (11.5,-2) circle (0.25cm);
    \draw [->, >=Stealth] (9.5,0) -- (11.25,-0.5);
    \draw [->, >=Stealth] (10.25,-0.75) -- (11.25,-1.25);
    \draw [->, >=Stealth] (10.25,-1.75) -- (11.25,-2);
    \draw [ dashed] (5,0.75) rectangle  (6,-3.5);
    \draw [ dashed] (6.5,0.75) rectangle  (10.5,-3.5);
    \draw [ dashed] (11,0.75) rectangle  (12,-3.5);
    \draw [->, >=Stealth] (5.75,-0.25) -- (6.75,-0.5);
    \draw [->, >=Stealth] (5.75,-1) -- (7.25,-1);
    \draw [->, >=Stealth] (5.75,-1.75) -- (7.5,-1.75);
    \draw [->, >=Stealth] (5.75,-2.5) -- (7,-2.25);
    \node [font=\normalsize] at (5.5,1) {Input Layer};
    \node [font=\normalsize] at (8.5,1) {Reservoir Layer};
    \node [font=\normalsize] at (11.5,1) {Output Layer};
    \end{circuitikz}
    }%
    \caption{Architecture of a reservoir computing system showing input layer, reservoir dynamics, and output layer.}
    \label{fig:reservoir_architecture}
\end{figure}

\subsection{Quantum Reservoir Computing}
Quantum Reservoir Computing (QRC) extends the reservoir 
computing paradigm into the quantum domain. By leveraging quantum systems' 
inherent properties, such as superposition and entanglement, QRC aims 
to enhance computational capabilities. Implementations using quantum 
oscillators have shown promise in solving complex learning tasks, 
offering advantages over classical counterparts. Notably, large-scale 
experiments utilizing neutral-atom analog quantum computers have demonstrated 
the scalability and effectiveness of QRC in various machine learning applications 
\cite{kornjavcaLargescaleQuantumReservoir2024}

In QRC, classical input data \( u(t) \) is encoded into quantum states \( |\psi(t)\rangle \),
which evolve under a fixed Hamiltonian \( H \):
\begin{equation}
    |\psi(t+1)\rangle = U |\psi(t)\rangle = e^{-iH\Delta t} |\psi(t)\rangle,
\end{equation}
where \( U \) is the unitary evolution operator. Measurements of observables \( \hat{O} \) yield outputs:
\begin{equation}
    y(t) = \langle \psi(t) | \hat{O} | \psi(t) \rangle.
\end{equation}
The output weights are trained classically, while the quantum reservoir remains fixed.

\subsection{Quantum Computing with Neutral Atoms}

% TALK ABOUT HAMILTONIAN


\subsection{Dimensionality Reduction for Image Data}
\subsubsection{Principal Component Analysis}
\subsubsection{Autoencoder Architectures}
\subsection{Quantum-Guided Autoencoding}

%============================================
% METHODOLOGY
%============================================
\section{Methodology}
\subsection{System Architecture Overview}
\subsection{Quantum Guided Autoencoder}
\subsubsection{Loss Function Design}
\subsubsection{Balancing Reconstruction and Classification}
\subsection{The Gradient Barrier Problem}
\subsection{Surrogate Modeling for Quantum Layers}
\subsubsection{Architecture and Training}
\subsubsection{Gradient Flow Through Surrogate Models}
\subsection{Rydberg Hamiltonian and Quantum Dynamics}
\subsection{Data Encoding Schemes}
\subsection{Quantum Readout Methods}
\subsubsection{Single-atom Measurements}
\subsubsection{Two-atom Correlations}
\subsubsection{Three-atom Correlations}

%============================================
% EXPERIMENTAL SETUP
%============================================
\section{Experimental Setup}
\subsection{Datasets}
\subsection{Implementation Details}
\subsubsection{Quantum Simulation Parameters}
\subsubsection{Classical Network Architectures}
\subsection{Comparison Methods}
\subsection{Performance Metrics}
\subsection{Parameter Sweep Strategy}

%============================================
% RESULTS
%============================================
\section{Results and Discussion}
\subsection{Classification Performance Comparison}
\subsection{Ablation Studies}
\subsubsection{Impact of Guided Lambda Parameter}
\subsubsection{Effect of Quantum Update Frequency}
\subsubsection{Influence of Quantum Parameters}
\subsection{Dimensionality Reduction Comparison}
\subsection{Surrogate Model Fidelity Analysis}
\subsection{Generalization to Unseen Data}

%============================================
% THEORETICAL ANALYSIS
%============================================
\section{Theoretical Analysis}
\subsection{Information Encoding in Quantum Reservoirs}
\subsection{Gradient Flow in Quantum-Classical Hybrid Systems}
\subsection{Computational Complexity}
\subsection{Quantum Resource Requirements}

%============================================
% LIMITATIONS AND FUTURE WORK
%============================================
\section{Limitations and Future Work}
\subsection{Current Limitations}
\subsection{Potential Extensions}
\subsection{Hardware Implementation Considerations}

%============================================
% CONCLUSION
%============================================
\section{Conclusion}

%============================================
% REFERENCES
%============================================
\bibliographystyle{IEEEtran}
\bibliography{references}

\end{document}